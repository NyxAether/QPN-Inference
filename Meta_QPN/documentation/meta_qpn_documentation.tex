\documentclass[11pt, a4paper]{article}

\usepackage[utf8]{inputenc}
\usepackage[T1]{fontenc}
\usepackage{ucs}
\usepackage{geometry}
\usepackage[pdftex]{graphicx}
\usepackage[english,francais]{babel}
\usepackage{setspace}
\usepackage{hyperref}
\usepackage{tabularx}
\usepackage[french]{varioref}
\usepackage{fancyhdr}
\usepackage{eurosym}
\usepackage{booktabs}
\usepackage{multirow}
\usepackage{color}
\usepackage[dvipsnames]{xcolor}
\usepackage{algorithmic}
\usepackage{algorithm}
\usepackage{tikz}
\usetikzlibrary{arrows,positioning,automata,shadows}
\usepackage{parskip}

\usepackage{amsmath,bm,times}
%\pagestyle{fancy}

%pour avancer ou reculer une date
\usepackage{advdate}
\newcommand{\advanceday}[1][-5]{%
\begingroup
\AdvanceDate[#1]%
\today%
\endgroup
}%


% \tikzset{
%      frame/.style={
%        rectangle, draw, 
%        text width=5em, text centered,
%        minimum height=4em,drop shadow,fill=white,
%        rounded corners,
%      },
%      line/.style={
%    	draw, -latex',rounded corners=2mm,
%      }
%}


\makeatletter
\newcommand\footnoteref[1]{\protected@xdef\@thefnmark{\ref{#1}}\@footnotemark}
\makeatother

\newcommand{\reporttitle}{Définitions et autres notes pour la lecture d'articles}     % Titre
\newcommand{\reportauthor}{Romain \textsc{Rincé}} % Auteur
\newcommand{\reportsubject}{} % Sujet
\newcommand{\HRule}{\rule{\linewidth}{0.3mm}}


\hypersetup{
    pdftitle={\reporttitle},%
    pdfauthor={\reportauthor},%
    pdfsubject={\reportsubject},%
    pdfkeywords={rapport} {vos} {mots} {clés},
    colorlinks=true,                % colorise les liens
    breaklinks=true,                % permet les retours à la ligne pour les liens trop longs
    urlcolor= blue,                 % couleur des hyperliens
    linkcolor= BrickRed,                % couleur des liens internes aux documents (index, figures, tableaux, equations,...)
    citecolor= green 
}

%Couverture 
\widowpenalty=10000
\clubpenalty=10000

%\setlength{\parskip}{1ex} % Espace entre les paragraphes
%\setlength{\parindent}{0pt}

\begin{document}


\begin{center}


%    \includegraphics [width=43mm]{UnivNantes.jpg} \\[3.2cm]
    %\textsc{Université de Nantes}


%\textsc{\Large \reportsubject}\\[0.4cm]
\HRule \\[0.3cm]
{\bfseries \reporttitle}\\[0.3cm]
\HRule \\[2ex]
\emph{Auteur : } \reportauthor
%\begin{minipage}[t]{0.3\textwidth}
%  \begin{flushleft} \large
%    \emph{Auteur :}\\
%    \reportauthor
%  \end{flushleft}
%\end{minipage}
%\begin{minipage}[t]{0.6\textwidth}
%  \begin{flushright} \large
%    \emph{Directeur de Recherche :} \\
%    Régnier \textsc{Pirard}
%   
%  \end{flushright}
%\end{minipage}
%
%\vfill
%
%{\large 2014-2015}

\end{center}

  
%\cleardoublepage
%\null

%\lhead{\reportauthor}
%\rhead{\reportsubject}
%\lfoot{Université de Nantes}
%\rfoot{2014-2015}
%	
%    
%    
%  \cleardoublepage % Dans le cas du recto verso, ajoute une page blanche si besoin
%  \tableofcontents % Table des matières
%  %\sloppy          % Justification moins stricte : des mots ne dépasseront pas des subsection*es
%
%
%
%\newpage

\section{Règles essentielles}

\subsection*{Règle de Bayes}\begin{equation}
P(x|y) = \frac{P(y|x).P(x)}{P(y)}
\end{equation}

\begin{equation}
P(x|y) = \frac{P(xy)}{P(y)}
\end{equation}

\subsection*{Opérateurs}

\begin{table}[h]
\centering
\begin{minipage}{0.4\textwidth}
\flushright

\begin{tabular}{c|cccc}
$\otimes$ & $+$ & $-$ & $0$ & ?\\
\hline
$+$ & $+$ & $-$ & $0$ & ?\\
$-$ & $-$ & $+$ & $0$ & ?\\
$0$ & $0$ & $0$ & $0$ & $0$\\
? & ? & ? & $0$ & ?\\

\end{tabular}
\end{minipage}\hspace{1em}
\begin{minipage}{0.4\textwidth}
\flushleft

\begin{tabular}{c|cccc}
$\oplus$ & $+$ & $-$ & $0$ & ?\\
\hline
$+$ & $+$ & ? & $+$ & ?\\
$-$ & ? & $-$ & $-$ & ?\\
$0$ & $+$ & $-$ & $0$ & ?\\
? & ? & ? & ? & ?\\

\end{tabular}
\end{minipage}
\end{table}

\subsection*{Influence}
\begin{align*}
&\forall x \in \cup_{C\in V(t)\backslash \lbrace A \rbrace } \pi_G(C) \backslash V(t)\\
&S_G^+(A,B,t) \iff P(b|ax)\geq P(b|\bar{a}x)
\end{align*}

\subsection*{Synergies}
\begin{align*}
\intertext{Pour un sous-graphe}
A \rightarrow C, B \rightarrow C
\intertext{On défini la \textbf{synergie additive} :}
&\forall x \in \pi_G(C) \backslash A,B\})\\
&Y_G^+(\lbrace A,B\rbrace ,C) \iff P(c|abx) + P(c|\bar{a}\bar{b}x) \geq P(c|a\bar{b}x + P(c|\bar{a}bx)
\end{align*}

\begin{align*}
\intertext{On défini la \textbf{synergie produit} :}
&\forall x \in \pi_G(C) \backslash A,B\})\\
&X_G^-(\lbrace A,B\rbrace ,C) \iff P(c_0|abx) . P(c_0|\bar{a}\bar{b}x) \leq P(c_0|a\bar{b}x . P(c_0|\bar{a}bx)
\end{align*}


\section{Choix des voisins pour la propagation}
Si A reçoit un message de B :
\begin{align*}
\intertext{Soit}
&G=(V_G, A_G)\\
&A,B\in V_G\\
&O \subseteq V_G\quad \text{les noeuds observés}\\
&X = \{X_i|X_i\in \sigma_G(A), \sigma_G^*(X_i)\cap O \neq \emptyset\}\quad \text{les enfants de A ayant un descendant observé}
\intertext{Les voisins de A sont définis par :}
&N = \left\{
\begin{array}{l l}
\sigma_G(A) \cup (\pi_G(X)\backslash\{A\}) & \text{if}\ B\rightarrow A\in A(G);\\
\pi_G(A) \cup (\sigma_G(A)\backslash \{B\})\cup (\pi_G(X)\backslash\{A\}) & \text{if}\ A \rightarrow B \in A(G).
\end{array} 
\right. \\
&\intertext{C'est à dire :}
&N = \left\{
\begin{array}{l l}
\text{Les fils de $A$ et leurs parents s'ils ont un descendant observé} & \text{if}\ B\rightarrow A\in A(G);\\
\text{Idem $+$ les parents de $A$} & \text{if}\ A \rightarrow B \in A(G).
\end{array} 
\right. \\
&\intertext{Les voisins \textbf{actifs} de A sont définis par $N\backslash O$}
\end{align*}

\section{Influences non-monotones}
Pour un graphe $A \rightarrow B, C\rightarrow B$
\begin{align*}
\intertext{On défini une \textbf{influence non-monotone} :}
&y,y' \in \pi_G(B) \backslash \{A\})\\
&P(b|ay) > P(b|\bar{a}y) \quad\text{et}\quad P(b|ay') < P(b|\bar{a}y')
\end{align*}
\paragraph{ Influences non-monotones à un actionneur}
\begin{align*}
S^{\sim_C}(A,B) \wedge Y^\delta(\{A,C\},B) \wedge C = c_i \implies S^{\delta \otimes sign[c_i]}(A,B)
\end{align*}
\paragraph{ Influences non-monotones à plusieurs actionneurs}
\begin{align*}
\intertext{La relation est généralisable pour plusieurs actionneurs $P_i \in P$:}
&S^{\sim_P}(A,B) \wedge \underset{P_i\in P}{\bigwedge} Y^{\delta_i}(\{A, P_i\},B) \wedge P = \underset{P_i\in P}{\bigwedge}p_{ij} \implies S^{\oplus_{P_i\in P}(\delta_i\otimes sign[p_{ij}])}(A,B)
\end{align*}

Lorsque un ou plusieurs actionneurs ne sont pas définie, l'influence propage le signe "\emph{?}".

\end{document}